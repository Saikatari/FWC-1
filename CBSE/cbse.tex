
\documentclass{article}                       
\usepackage{siunitx}                                              
\usepackage{setspace}
\usepackage{gensymb}                                              
\usepackage{xcolor}                                               
\usepackage{caption}
%\usepackage{subcaption}
\doublespacing                                                   
\singlespacing                                                    
\usepackage[none]{hyphenat}
\usepackage{amssymb}
\usepackage{relsize}
\usepackage[cmex10]{amsmath}
\usepackage{mathtools}
\usepackage{amsmath}                                              
\usepackage{commath}                                              
\usepackage{amsthm}
\interdisplaylinepenalty=2500
%\savesymbol{iint}
\usepackage{txfonts}                                              
%\restoresymbol{TXF}{iint}                                        
\usepackage{wasysym}                                              
\usepackage{amsthm}
\usepackage{mathrsfs}                                             
\usepackage{txfonts}                                              
\let\vec\mathbf{}
\usepackage{stfloats}
\usepackage{float}
\usepackage{cite}
\usepackage{cases}                                                
\usepackage{subfig}                                               
%\usepackage{xtab}
\usepackage{longtable}
\usepackage{multirow}
%\usepackage{algorithm}
\usepackage{amssymb}
%\usepackage{algpseudocode}
\usepackage{enumitem}
\usepackage{mathtools}
%\usepackage{eenrc}
%\usepackage[framemethod=tikz]{mdframed}                          
\usepackage{listings}                                             
%\usepackage{listings}
\usepackage[latin1]{inputenc}
%%\usepackage{color}{
%%\usepackage{lscape}
\usepackage{textcomp}
\usepackage{titling}
\usepackage{hyperref}
%\usepackage{fulbigskip}
\usepackage{tikz}
\usepackage{graphicx}                                             
\lstset{
  frame=single,
  breaklines=true
}
\let\vec\mathbf{}
\usepackage{enumitem}                                             
\usepackage{graphicx}                                             
\usepackage{siunitx}
\let\vec\mathbf{}                                                
\usepackage{enumitem}
\usepackage{graphicx}
\usepackage{enumitem}
\usepackage{tfrupee}
\usepackage{amsmath}
\usepackage{amssymb}
\usepackage{mwe} % for blindtext and example-image-a in example
\usepackage{wrapfig}
\usepackage{enumitem}
\providecommand{\qfunc}[1]{\ensuremath{Q\left(#1\right)}}
\providecommand{\sbrak}[1]{\ensuremath{{}\left[#1\right]}}
\providecommand{\brak}[1]{\ensuremath{\left(#1\right)}}
\providecommand{\lbrak}[1]{\ensuremath{\left(#1\right.}}
\providecommand{\rbrak}[1]{\ensuremath{\left.#1\right)}}
\providecommand{\cbrak}[1]{\ensuremath{\left\{#1\right\}}}
\providecommand{\lcbrak}[1]{\ensuremath{\left\{#1\right.}}
\providecommand{\rcbrak}[1]{\ensuremath{\left.#1\right\}}}
\newcommand{\mydet}[1]{\ensuremath{\begin{vmatrix}#1\end{vmatrix}}}
\newcommand{\myvec}[1]{\ensuremath{\begin{bmatrix}#1\end{bmatrix}}}
\begin{document}
\section{vectors}
\begin{enumerate}	
\item Find the angle between the line $\overrightarrow{r} = \brak{2\hat{i}-\hat{j}+3\hat{k}} + \lambda\brak{3\hat{i}-\hat{j}+2\hat{k}}$ and the  plane $\overrightarrow{r}.\brak{\hat{i}+\hat{j}+\hat{k}}=3$.	
\item Using vectors, prove that the points $\brak{2, -1, 3}, \brak{3, -5, 1}$ and $\brak{-1, 11, 9}$ are collinear.
\item For any two vectors $\overrightarrow{a}$ and $\overrightarrow{b}$, prove that $\brak{\overrightarrow{a} \times \overrightarrow{b}}^{2}=\overrightarrow{a}^{2}\overrightarrow{b}^{2}-\brak{\overrightarrow{a} . \overrightarrow{b}}^{2}$
\item Using vectors, find the value of X such that the four points A$\brak{X, 5, -1}$, B$\brak{3, 2, 1}$, C$\brak{4, 5, 5}$ and D$\brak{4, 2, -2}$ are coplanar.
\end{enumerate}
\section{Linear Forms}
\begin{enumerate}
\item Find the co-ordinates of the point, where the line ${\frac {X+2}{1}}={\frac{y-5}{3}}={\frac{z+1}{5}}$ cuts the yz-plane.
\item Find the equation of the plane passing through the point $\brak{-1, 3, 2}$ and perpendicular to the planes $x+2y+3z=5$ and $3x+3y+z=0$.
\end{enumerate}
\section{probability}
\begin{enumerate}
\item Four cards are drawn one by one with replacement from a well-shuffled deck of playing cards.Find the probability that at least three cards are of diamonds.
\item The probbility of two students A and B coming to school on time are $\frac{2}{7}$ and $\frac{4}{7}$, respectively.Assuming that the events 'A coming on time' and 'B coming on time' are independent, find the probability of only one of them coming to school on time.
\item If A and B are independent events with P$\brak{A}=\frac{3}{7}$ and P$\brak{B}=\frac{2}{5}$, then find P$\brak{A^{\prime} \cap B^{\prime}}$.
\end{enumerate}
\section{optimization}
\begin{enumerate}
\item A company manufactures two types of novelty souvenirs made of plywood. Souvenirs of type A require 5 minutes each for cutting and 10 mintues each for assembling. souvenirs of type B require 8 minutes each for cutting and assemblig. The profit for type A souvenirs is \rupee 100  each and for type B souvenirs, profit is \rupee 120  each. How many souvenirs of each type should the company manufacture in order to maximise the profit? Formulate the problem as a LPP and then slove it graphically.
\end{enumerate}
\section{differentiation}
\begin{enumerate}
\item Find the differential equation representing the family of curves $y=-A\cos3x+B\sin3x$.
\item Find the differential of the function $\cos^{-1}\brak{\sin2x}$ w.r.t.x.
\item Slove the following differential equation: 
		$\brak{y+3x^{2}}\frac{dx}{dy}=x$.
\item Differentiate $\tan^{-1}\frac{3x-x^{3}}{1-3x^{2}}$, $\mydet{x}<\frac{1}{\sqrt{3}}$ w.r.t $\tan^{-1}\frac{x}{\sqrt{1-x^{2}}}$.
\item If $\sqrt{1-x^{2}}+\sqrt{1-y^{2}}=a\brak{x-y}, \mydet{x}<1, \mydet{y}<1$, show that $\frac{dx}{dy}=\sqrt{\frac{1-y^{2}}{1-x^{2}}}$.
\item Find the particular solution of the differential equation:
	$\brak{1+e^{2x}}dy+\brak{1+y^{2}}e^{x}dx=0$, given that y$\brak{0}=1$.
\item Find the particular solution of the differential equation:
	$x\dfrac{dy}{dx}\sin \brak{\dfrac{y}{x}}+x-y\sin\brak{\dfrac{y}{x}}=0$, given that $y\brak{1}=\dfrac{\pi}{2}$.
\item If $y=\brak{\sin x}^{x}+\sin^{-1}\brak{\sqrt{1-x^{2}}}$, then find $\dfrac{dy}{dx}$.
\item Find the interval inwhich the funnction f given by $f\brak{x}=\sin 2x+\cos 2x, 0\leq x\leq \pi$ is strictly decreasing.
\end{enumerate}
\section{Integration}
\begin{enumerate}
\item Find:

		$\int\frac{x-1}{\brak{x-2}\brak{x-3}}dx$
\item integrate:

		$\frac{e^{x}}{\sqrt{5-4e^{x}-e^{2x}}}$ with respect to x.
\item Find:
$\int e^{x}\brak{\dfrac{2+\sin 2x}{2\cos^{2}x}} dx$
\item Evaluate:
	$\int_{1}^5 \brak{\mydet{x-1}+\mydet{x-2}+\mydet{x-4}}dx$
\item Find:\begin{align*}\int \cos2x\cos4x\cos6x dx \end{align*}
\item Using integration, find the area of the following region:$\cbrak{\brak{x, y}:x^{2}+y^{2} \leq16a^{2} \text {and} y^{2}\leq6ax}$
\item Using integration, find the area of area of triangle ABC bounded by the lines $ 4x-y+5=0,x+y-5=0 \text {and} x-4y+5=0$.
\end{enumerate}
\section{Function}
\begin{enumerate}
\item If an operation $\ast$ on the set of integers Z is defined $a \ast b = 2a^{2}+b$, then find \brak{i} whether it is a binary or not, and \brak{ii} if a binary, then is it commutative or not.
\item prove that:

	$\sin^{-1}\frac{4}{5}+\tan^{-1}\frac{5}{12}+\cos^{-1}\frac{63}{65}=\frac{\pi}{2}$
\item Prove that the relation R in the set A=$\cbrak{1, 2, 3, 4, 5, 6, 7}$ given by R=\cbrak{\brak{a, b}:\mydet{a-b} is   even} is an equivalence relation.
\item Show that the function f in A=R- $\cbrak{\frac{2}{3}}$ defined as f\brak{x}=$\frac{4x+3}{6x-4}$ is one-one and onto. Hence, find $f^{-1}$.
\end{enumerate}
\section{Matrices}
\begin{enumerate}
\item If A is a square matrix of order $2$ and \mydet{A}=$4$, then find the value of $\mydet{2.A A^{\prime}}$, where $A^{\prime}$ is the transpose of matrix A.
\item Find the value of \brak{x-y} from the matrix equation $2\myvec{x & 5\\7 & y-3}+\myvec{-3&-4\\1&2}=\myvec{7&6\\15&14}$.
\item prove that:

	$\sin^{-1}\frac{4}{5}+\tan^{-1}\frac{5}{12}+\cos^{-1}\frac{63}{65}=\frac{\pi}{2}$
\item Using elementary row transformations, find the inverse of the matrix \myvec{3&0&-1\\2&3&0\\0&4&1}.
\item Using matrices, solve the following system of linear equations:\begin{align*}2x+3y+10z=4\\4x-6y+5z=1\\6x+9y-20z=2 \end{align*}
\end{enumerate}
\end{document}
