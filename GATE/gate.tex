
\documentclass[12pt,-letter paper]{article}                       
\usepackage{siunitx}                                              
\usepackage{setspace}
\usepackage{gensymb} 
\usepackage{circuitikz}                       
\usepackage{xcolor}                                               
\usepackage{caption}
%\usepackage{subcaption}
\doublespacing                                                    
\singlespacing                                                    
\usepackage[none]{hyphenat}
\usepackage{amssymb}
\usepackage{relsize}
\usepackage[cmex10]{amsmath}
\usepackage{mathtools}
\usepackage{amsmath}                                              
\usepackage{commath}                                              
\usepackage{amsthm}
\interdisplaylinepenalty=2500
%\savesymbol{iint}
\usepackage{txfonts}                                              
%\restoresymbol{TXF}{iint}                                        
\usepackage{wasysym}                                              
\usepackage{amsthm}
\usepackage{mathrsfs}                                             
\usepackage{txfonts}                                              
\let\vec\mathbf{}
\usepackage{stfloats}
\usepackage{float}
\usepackage{cite}
\usepackage{cases}                                                
\usepackage{subfig}                                               
%\usepackage{xtab}
\usepackage{longtable}
\usepackage{multirow}
%\usepackage{algorithm}
\usepackage{amssymb}
%\usepackage{algpseudocode}
\usepackage{enumitem}
\usepackage{mathtools}
%\usepackage{eenrc}
%\usepackage[framemethod=tikz]{mdframed}                          
\usepackage{listings}                                             
%\usepackage{listings}
\usepackage[latin1]{inputenc}
%%\usepackage{color}{
%%\usepackage{lscape}
\usepackage{textcomp}
\usepackage{titling}
\usepackage{hyperref}
%\usepackage{fulbigskip}
\usepackage{tikz}
\usepackage{graphicx}                                             
\lstset{
  frame=single,
  breaklines=true
}
\let\vec\mathbf{}
\usepackage{enumitem}                                             
\usepackage{graphicx}                                             
\usepackage{siunitx}
\let\vec\mathbf{}                                                 
\usepackage{enumitem}
\usepackage{graphicx}
\usepackage{enumitem}
\usepackage{tfrupee}
\usepackage{amsmath}
\usepackage{amssymb}
\usepackage{mwe} % for blindtext and example-image-a in example
\usepackage{wrapfig}
\title{Matrix assignment}
\date{\today}
\usepackage{graphics}
\providecommand{\brak}[1]{\ensuremath{\left(#1\right)}}
%\title{}
\author{Prof.G V V sharma}
\usepackage{graphicx}
\graphicspath{{/storage/emulated/o/Fwc/gate/}}
\usepackage{wrapfig}
\begin{document}
\title{\textbf{GATE}}
\begin{enumerate}
	\item A four-variable boolean function is realized using $4x1$ multiplexers as shown in the figure.
\begin{figure}[H]
\centering
	
\scalebox{0.7}{
\fontsize{18pt}{20pt}\selectfont
\begin{circuitikz}

\draw
(0,0) -- (0,5) -- (3,5) -- (3,0) -- cycle
(6,0) -- (6,5) -- (9,5) -- (9,0) -- cycle
(-0.1,1) node[] { $\blacktriangleright$} node[anchor=west] {$I_{3}$} -- (-2,1) -- (-2,4) -- (-0.1,4) node[] { $\blacktriangleright$} node[anchor=west] {$I_{0}$}
(-0.1,2)node[] { $\blacktriangleright$} node[anchor=west] {$I_{2}$} -- (-0.5,2) -- (-0.5,3) -- (-0.1,3) node[] { $\blacktriangleright$} node[anchor=west] {$I_{1}$}
(3,4) -- (5.9,4) node[] { $\blacktriangleright$} node[anchor=west] {$I_{0}$}
(4.5,3) -- (4.5,4) -- (4.5,3) -- (5.9,3) node[] { $\blacktriangleright$} node[anchor=west] {$I_{1}$}
(5,-1) node[ground] {} -- (5,2) -- (5.9,2) node[] { $\blacktriangleright$} node[anchor=west] {$I_{2}$}
(5,1) -- (5.9,1) node[] { $\blacktriangleright$} node[anchor=west] {$I_{3}$}
(9,4) -- (10,4) node[] { $\blacktriangleright$} node[anchor=west] {F(U,V,W,X)}
(1.5,0) node[anchor=south] {$S_{1}$} -- (1.5,-0.8) node[anchor=north] {U}
(2.5,0) node[anchor=south] {$S_{0}$} -- (2.5,-0.8) node[anchor=north] {V}
(-2,1) -- (-2,-1) node[ground] {}
(7.5,0)node[anchor=south] {$S_{1}$} -- (7.5,-0.8) node[anchor=north] {W}
(8.5,0)node[anchor=south] {$S_{0}$} -- (8.5,-0.8) node[anchor=north] {X}
(-0.5,2.5) -- (-1,2.5) node[anchor=east] {Vcc}
(1,3) node[anchor=west]{4x1}
(1,2.5) node[anchor=west]{MUX}
(7,3) node[anchor=west]{4x1}
(7,2.5) node[anchor=west]{MUX}
;
\end{circuitikz}
}

		\caption{four-variable boolean function}
		\label{fig}
	\end{figure}
		\hfill(GATE EC 2018)


 The minimized expression for $F(U,V,W,X)$ is
\begin{enumerate}
\item $\brak{UV+ \bar{U}\bar{V}}\bar{W}$
\item $\brak{UV+ \bar{U}\bar{V}}\brak{\bar{W}\bar{X}+\bar{W}X}$
\item $ \brak{U\bar{V}+\bar{U}V}\bar{W}$
\item $\brak{U\bar{V}+\bar{U}V}\brak{\bar{W}\bar{X}+\bar{W}X}$
\end{enumerate}
\end{enumerate}
\end{document} 
